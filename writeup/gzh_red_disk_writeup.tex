\documentclass[useAMS,usenatbib]{mn2e}
%\documentclass[twocolumn]{emulateapj}
\usepackage{graphicx,natbib,color,multirow,amsmath,url,soul}
\usepackage{epsfig}
\usepackage{float}
\usepackage{deluxetable}

\newcommand\ion[2]{[#1$\;${\scshape{#2}}]}      % ion, i.e., [CII] = \ion{C}{ii}
\newcommand\pfeatures{$p_{\rm{features~or~disk}}$}
\newcommand\plusminus[2]{\genfrac{}{}{0pt}{}{#1}{#2}}
\newcommand\pnotedgeon{$p_{\rm{not~edge-on}}$}
\newcommand\pbar{$p_{\rm{bar}}$}
\newcommand\pnobar{$p_{\rm{no~bar}}$}
\newcommand\gztwo{Galaxy~Zoo~2}
\newcommand\mbh{$M_{\rm{BH}}$}
\newcommand\db{$d_{\rm{B-NB}}$}
\newcommand\fb{$f_{\rm{B>NB}}$}
\newcommand\pasa{PASA}
\voffset-1.25cm
\begin{document}

\title[Galaxy~Zoo: passive disk fraction]{Galaxy~Zoo~Hubble: the passive disk fraction decreases from $z=1.0$ to $z=0.3$. \emph(working title)}
\author[Galloway et~al.]{\parbox[t]{16cm}{Melanie A. Galloway$^1$, Kyle W. Willett$^1$,several others
\vspace{0.1in} }\\
$^{1}$School of Physics and Astronomy, University of Minnesota, 116 Church St. SE, Minneapolis, MN 55455, USA\\
$^{2}$Wheelock College, Department of Science, Wheelock College, Boston, MA 02215, USA\\
$^{3}$Institute for Astronomy, Department of Physics, ETH Z\"urich, Wolfgang-Pauli-Strasse 16, CH-8093, Z\"urich, Switzerland\\
$^{4}$Department of Astronomy and Astrophysics, 1156 High Street, University of California, Santa Cruz, CA 95064, USA\\
$^{5}$Kavli IPMU (WPI), The University of Tokyo, Kashiwa, Chiba 277-8583, Japan\\
$^{6}$Oxford Astrophysics, Denys Wilkinson Building, Keble Road, Oxford OX1 3RH, UK\\
$^{7}$Institute of Cosmology \& Gravitation, University of Portsmouth, Dennis Sciama Building, Portsmouth PO1 3FX, UK\\
$^{8}$SEPnet --- \url{http://www.sepnet.ac.uk} \\
   }
\maketitle

\begin{abstract}


\end{abstract}

\section{Introduction}
\label{sec:Intro}

\section{Data and sample selection}
\label{sec:Sample Selection}


\section{Results}
\label{sec:Results}


\section{Discussion}\label{sec:Discussion}


\section{Conclusions}
\label{sec:conclusions}



%%%%%%%%%%%%
%%% ACKNOWLEDGMENTS
%%%%%%%%%%%%

\section*{Acknowledgments}
The data in this paper are the result of the efforts of the Galaxy~Zoo~Hubble volunteers, without whom none of this work would be possible. Their efforts are individually acknowledged at \url{authors.galaxyzoo.org}. Please contact the author(s) to request access to research materials discussed in this paper. 


Since I'm from physics I've never taken Astro 101, but for those of you who did I believe you would have learned about this color magnitude diagram. You probably learned that when you plot a bunch of galaxies in this way, you get two distinct populations: in the bottom half of the diagram are the very blue galaxies, we call this the 'blue cloud.' When you look at the morphologies of the blue cloud, you find that almost all of them are spirals, or have disk structure. On the top you get the redder galaxies, we call this region the red sequence. The vast majority of these are ellipticals. 

You probably also learned in astro 101 that the color of a galaxy is an indicator of its age: blue light tends to come from young, hot stars, or from star-forming regions, while red light comes from old, cooler stars. This relationship between color and morphology, or similarly age and morphology, seems to imply some sort of morphological transformation as a galaxy ages - so we tend to believe that galaxies form as disks and then transform into ellipticals either soon after or before, or while the stars die.

So we have this model in which disks seem to transition into ellipticals around the same time as they stop forming stars, but this diagram doesn't tell us much about that process. Is it immediate - Does quenching happen for the same reason as the morphological tranformation? If not, does one happen before the other, and if so, which, and how? 

The way we can approach this possible intermediate phase in a galaxy's evolution is to study the red disks - the fact that they exist at all seems to imply that the morphological transformation is (at least not always) in sync with SF quenching; if it were, then we would *only* see BD and RE. Since there do exist small populations of RD and BEs, it is possible that they represent short stages within this larger evolutionary pathway. 

So what I'm going to talk to you about today is my research on these red disk galaxies, and how their population (in terms of fraction) changes over time (specifically between redshift 0.3 and 1.0), and how I'm going to try to argue how the fact that it changes at all, and whether it increases or decreases reveals details on the likely phases a typical galaxy goes through in its lifetime. But first I'm going to give some detail on the data set and how I selected these galaxies. 

This research made extensive use of the Tool for OPerations on Catalogues And Tables (TOPCAT), which can be found at \url{www.starlink.ac.uk/topcat/} \citep{tay05}. 



\bibliographystyle{mn2e}
\bibliography{melrefs}  

\end{document}
